% !mode::"TEX:UTF-8"
\documentclass{article}
\usepackage{CJKutf8}
\usepackage{indentfirst}      % 设置首行缩进
\setlength{\parindent}{2em}   % 缩进2个字符

\author{xhj}
\begin{CJK}{UTF8}{gkai}
\title{PCA算法}
\end{CJK}

\begin{document}
\begin{CJK}{UTF8}{gkai}    % 在pdf中能够正确显示中文
\maketitle

%===================first section===================
\section{最近重构性}
最近重构性要求样本点到降维后的超平面的距离之和最短,以此来确定该超平面。

%===================================================
%===================second section==================
\section{最大可分性}
\subsection{基于特征值和特征向量求主成分}
1、中心化(均值化)。目的是为了方便之后的求解。因为需要求解协方差矩阵,协方差的公式为$cov(x,y)=E\{(x-Ex)(y-Ey)\}$。利用中心化之后的矩阵就可以直接求协方差。\\

2、求协方差矩阵。协方差矩阵的维数由数据的特征的数目决定。
\begin{equation}
cov=
\left(
\begin{array}{cccc}
cov(x_0,x_0) & cov(x_0,x_1) & ... & cov(x_0,x_n)\\
cov(x_1,x_0) & cov(x_1,x_1) & ... & cov(x_1,x_n)\\
... & ... & ... & ...\\
cov(x_n,x_0) & cov(x_n,x_1) & ... & cov(x_n,x_n)
\end{array}
\right)
\end{equation}
通过求解协方差矩阵,可以观察样本中各个特征之间的相关度。通过协方差也可以求解相关系数$\rho$,$\rho=\frac{cov(x,y)}{\sqrt{D(x)D(y)}}$。\\

3、求解协方差矩阵的特征值和特征向量。在协方差矩阵的特征值中选取最大的k个,并且提取它们对应的特征向量。此步骤的目的就是构建一个正交的新的坐标系,因为特征向量都是正交的特征向量,所以
%===================================================
\end{CJK}
\end{document}